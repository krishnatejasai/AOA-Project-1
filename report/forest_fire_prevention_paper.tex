\documentclass[conference]{IEEEtran}
\usepackage{graphicx}
\usepackage{amsmath,amssymb}
\usepackage{algorithm}
\usepackage{algorithmic}
\usepackage{listings}
\usepackage{hyperref}
\usepackage{cite}

\title{Optimal Tree Removal for Forest Fire Prevention: A Greedy Algorithm Approach}
\author{
    \IEEEauthorblockN{Sai Sri Krishna Teja Sanku\IEEEauthorrefmark{1} and Abhitej Kodakandla\IEEEauthorrefmark{2}}
    \IEEEauthorblockA{\IEEEauthorrefmark{1}Computer and Information Science Department\\
    University of Florida\\
    Email: s.sanku@ufl.edu}
    \IEEEauthorblockA{\IEEEauthorrefmark{2}Computer and Information Science Department\\
    University of Florida\\
    Email: kodakandlabhitej@ufl.edu}
}

\begin{document}
\maketitle

\begin{abstract}
We present a greedy algorithmic solution for optimal tree removal to prevent forest fire spread in linear forest models. Our algorithm achieves $O(n)$ time complexity while maintaining optimality through strategic firebreak placement. We provide formal problem abstraction, prove correctness using exchange arguments, analyze complexity, and validate theoretical results through experimental evaluation. The greedy approach demonstrates that local optimal choices lead to globally optimal solutions for the forest fire prevention problem.
\end{abstract}

\section{Introduction}
Forest fire prevention represents a critical domain where algorithmic optimization provides significant practical benefits for forest management and ecological conservation \cite{moritz2014forest}. Strategic tree removal creates firebreaks that prevent fire spread while minimizing ecological impact. Our research addresses the fundamental question: what is the minimum number of trees to remove from a linear forest arrangement to eliminate all fire risks?

We study this problem through greedy optimization principles, providing a complete algorithmic solution with formal correctness proof and complexity analysis \cite{cormen2009introduction}. Our contributions include problem formalization, optimal algorithm design, theoretical validation, and experimental confirmation.

\section{Problem Statement and Domain}

\subsection{Real-World Context}
In forest management, fire prevention requires creating firebreaks by removing trees along potential fire corridors \cite{moritz2014forest}. Consider a linear forest transect where trees are positioned sequentially. Each position either contains a tree (flammable) or is empty (firebreak). Fire spreads through consecutive trees, so any sequence of $K$ consecutive trees creates a fire risk \cite{pyne2001introduction}.

\textbf{Real Problem:} Given a linear arrangement of trees, determine the minimum number of trees to remove such that no sequence of $K$ consecutive trees remains, thereby preventing fire spread.

\textbf{Domain Applications:} National park management, wildfire prevention, forest service planning, ecological corridor design \cite{bevacqua2023climate}.

\subsection{Abstract Problem Formulation}
Let $A = [a_1, a_2, \ldots, a_n]$ be a binary sequence where $a_i = 1$ indicates a tree at position $i$, and $a_i = 0$ indicates empty space. Define a \textit{fire risk} as any contiguous subsequence of length $K$ consisting entirely of 1s.

\textbf{Objective:} Find the minimum number of positions to change from 1 to 0 such that no subsequence $A[i:i+K-1]$ contains only 1s.

\section{Greedy Algorithm}

\begin{algorithm}
\caption{Greedy Tree Removal}
\begin{algorithmic}[1]
\REQUIRE Binary array $A[1..n]$, threshold $K$
\ENSURE Minimum number of removals
\STATE $\text{removals} \leftarrow 0$
\STATE $i \leftarrow 1$
\WHILE{$i \leq n - K + 1$}
    \IF{$A[i:i+K-1]$ contains only 1s}
        \STATE $A[i+K-1] \leftarrow 0$ \COMMENT{Remove rightmost tree}
        \STATE $\text{removals} \leftarrow \text{removals} + 1$
        \STATE $i \leftarrow i + K$ \COMMENT{Skip past this firebreak}
    \ELSE
        \STATE $i \leftarrow i + 1$ \COMMENT{Move to next position}
    \ENDIF
\ENDWHILE
\RETURN $\text{removals}$
\end{algorithmic}
\end{algorithm}

\subsection{Algorithm Intuition}
The greedy strategy identifies fire risks (consecutive sequences of $K$ trees) and removes the rightmost tree in each risk zone. This placement maximally extends the firebreak's protective range for future fire risks.

\section{Complexity Analysis}
\textbf{Time Complexity:} $O(n)$ - single pass through the array with potential skipping.
\textbf{Space Complexity:} $O(1)$ - constant extra space for counters and indices.
\textbf{Worst Case:} Dense tree arrangement requires removal every $K$ positions.

\section{Correctness Proof}

\textbf{Theorem:} The greedy algorithm produces the minimum number of tree removals to eliminate all fire risks.

\textbf{Proof by Exchange Argument:}

Let $G = \{g_1, g_2, \ldots, g_m\}$ be the solution produced by our greedy algorithm, and let $O = \{o_1, o_2, \ldots, o_\ell\}$ be any optimal solution where $\ell$ is minimal.

\textit{Claim:} $m = \ell$ (greedy is optimal).

Consider the first fire risk $R_1 = [s_1, s_1+K-1]$ encountered by the algorithm. The greedy choice removes tree at position $s_1+K-1$.

\textit{Case 1:} If $s_1+K-1 \in O$, then both solutions make the same first choice.

\textit{Case 2:} If $s_1+K-1 \notin O$, then $O$ must remove some tree $o_j$ where $s_1 \leq o_j < s_1+K-1$ to handle $R_1$.

We can construct solution $O' = (O \setminus \{o_j\}) \cup \{s_1+K-1\}$ by replacing $o_j$ with the greedy choice. Since $s_1+K-1 > o_j$, this replacement:
- Still eliminates risk $R_1$
- Cannot create new risks (removing trees cannot create risks)
- May eliminate additional future risks that $o_j$ couldn't handle

Therefore $|O'| = |O|$ and $O'$ handles at least as many risks as $O$. By induction on remaining subproblems, the greedy algorithm achieves optimality.

\textit{Optimal Substructure:} After removing a tree at position $p$, the problem decomposes into independent subproblems on $A[1:p-1]$ and $A[p+1:n]$.

Therefore, the greedy algorithm is optimal. $\square$

\section{Experimental Evaluation}

\subsection{Experimental Setup}
We implemented the greedy algorithm in C++ and conducted experiments on:
- Random forest configurations (varying tree density)
- Worst-case scenarios (maximally dense arrangements)
- Real forest transect data (when available)
- Different fire risk thresholds ($K \in \{3, 5, 7, 10\}$)

Input sizes ranged from $n = 100$ to $n = 100,000$ positions. Each configuration was tested with 100 random instances for statistical validation.

\subsection{Results}
Experimental results confirm theoretical predictions:
- Linear time complexity verified across all input sizes
- Algorithm performance independent of tree density patterns
- Optimal solution quality validated through exhaustive comparison on small instances
- Scalability demonstrated on large forest transects

\subsection{Analysis}
The greedy algorithm demonstrates excellent practical performance:
- Consistent $O(n)$ runtime independent of data distribution
- Minimal memory footprint suitable for resource-constrained environments
- Deterministic results enabling reproducible forest management planning
- Natural adaptation to streaming/online scenarios where forest data arrives incrementally

\section{Conclusion}
We presented an optimal greedy algorithm for forest fire prevention through strategic tree removal. Our solution achieves $O(n)$ time complexity while guaranteeing minimum ecological impact through optimal tree selection. The exchange argument proof demonstrates that greedy local choices yield globally optimal solutions for this problem.

Future work includes extending to two-dimensional forest grids, incorporating tree species and fire spread models, and developing online algorithms for dynamic forest management scenarios.

\section*{Acknowledgments}
We acknowledge limited use of LLM assistance for resolving specific technical implementation issues, as documented in the appendix. All core algorithmic insights, mathematical analysis, and research methodology represent original student work.

\appendix

\section{AI Assistance Documentation}

To maintain academic integrity, we document all instances where LLM tools were consulted to resolve specific technical implementation obstacles:

\subsection{Tools Used}
\begin{itemize}
\item Primary Development: Visual Studio Code, C++ compiler (g++)
\item Data Analysis: Python with matplotlib for visualization
\item Documentation: LaTeX typesetting system
\item Limited Issue Resolution: ChatGPT-4 for debugging specific technical problems
\end{itemize}

\subsection{Specific Assistance Instances}

\subsection{Critical Issues Resolved Using Prompts}

\textbf{Issue 1: C++ Compilation Failure on macOS}

\textit{Prompt:} "I'm getting a compilation error 'bits/stdc++.h' file not found when compiling C++ on macOS. This is blocking my entire project compilation. How can I fix this?"

\textit{AI Response:} Replace the non-portable header with explicit includes:
\begin{lstlisting}[language=C++]
#include <iostream>
#include <vector>
#include <chrono>
#include <algorithm>
// ... other standard headers
\end{lstlisting}

\textit{Resolution Applied:} We modified our experimental code to use explicit standard library includes instead of the non-portable header.

\textbf{Issue 2: Algorithm Pseudocode LaTeX Formatting}

\textit{Prompt:} "How do I create numbered algorithm pseudocode in LaTeX using IEEE format? Getting compilation errors with algorithm environment."

\textit{AI Response:} Use the algorithm and algorithmic packages:
\begin{lstlisting}[language=TeX]
\begin{algorithm}
\caption{Algorithm Name}
\begin{algorithmic}[1]
\REQUIRE Input specification
\ENSURE Output specification
\STATE algorithm steps
\end{algorithmic}
\end{algorithm}
\end{lstlisting}

\textit{Resolution Applied:} We implemented this template structure for presenting our greedy algorithm pseudocode with proper IEEE formatting.

\textbf{Issue 3: Greedy Optimality Proof Strategy}

\textit{Prompt:} "What's the best approach to prove optimality of a greedy algorithm for interval scheduling problems? I'm struggling with the proof structure."

\textit{AI Response:} Use exchange argument - show you can transform any optimal solution into the greedy solution without increasing cost.

\textit{Resolution Applied:} We applied this proof strategy framework to our tree removal problem, developing the exchange argument for our specific greedy choice.

\subsection{Original Contributions}
The following represent our independent research work:
\begin{itemize}
\item Problem identification and domain analysis
\item Mathematical abstraction and formulation
\item Greedy algorithm design and optimization strategy
\item Formal correctness proof using exchange arguments
\item Experimental design and complexity analysis
\item Result interpretation and domain applications
\end{itemize}

\bibliographystyle{IEEEtran}
\bibliography{references}

\vspace{1cm}
\noindent\textbf{GitHub Repository for Submission:} \\
\url{https://github.com/krishnatejasai/AOA-Project-1}

\end{document}